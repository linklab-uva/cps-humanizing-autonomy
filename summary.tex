%NSF-CPS

Autonomous cars are not just about the technology. They are about freedom of mobility, and a whole set of experiences that will literally and figuratively move people in new ways.
While the promise of self-driving cars is attractive, applying it in a meaningful and coherent way still remains a major challenge. Recent accidents involving autonomous (or semi-autonomous) cars that resulted in fatalities of either the driver, or in one case the pedestrian, have exposed major holes regarding the interaction between the car and its driver/passengers. We know now, that it may not always be possible for the driver to take over the control from the autonomous vehicle (AV) at any stage.Therefore, before it’s too late, we need to reevaluate our approach towards autonomy, by seeking answers to questions that put humans at the center stage. As journeys become fully automated, the experience itself will need to become more human. 

The goal of the proposed research is to utilize both human and machine advantages to humanize autonomy by instilling the beneficial nuance of human behavior and trust, while exploiting technological and safety benefits of an autonomous vehicle. The proposed research aims to bring human factors such as trust, and emotional behavior into the autonomous loop. Trust is one of the core human needs that the autonomous vehicle must establish and defend if the technology is to be adopted at all. However, the nature of trust in the vehicle or the autonomous system varies from person to person. In this research, we will run experiments on a full-scale driving simulator, and real in real vehicles, to understand perspectives of trust in autonomous vehicles from real people, rather than working on assumptions. Transparency and communication are critical to building trust. To establish user understanding of the system and its capabilities the interface must communicate clearly, and transparently - by revealing what the car sees, what the system is currently doing, what it intends to do in response to environmental conditions and why. 
Safety is not primarily just a functional consideration, it is also emotional. We propose that the issues of functional and emotional design for autonomous vehicles should be tackled together.
For example, emotional down-regulation could be used when passengers might be facing an upsetting or frustrating situation – for instance, a delay in travel. Here the AV could sense the frustration and then down-regulate through voice prompts. When you’re jumping in and out of different AVs, a consistent and personal experience will be vital for successful adoption.
 We’re concerned with a person’s ability, and even right, to make their own decisions, and come and go as they please. Not about how clever cars are without human drivers. 

\noindent\emph{\textbf{Intellectual Merit:}} The contributions of this proposal are as follows:
\begin{enumerate}
    \item Behavior guided autonomy \arsalan{Arsalan add} \lu{Lu add}
    \item Reachability analysis and control synthesis for safety region estimation \nicola{Nicola to add brief description.}
    \item Feedback design: We propose building a framework, that provides both the intent of the autonomous car and an explanation for its behavior to the driver/passenger by augmenting existing user interfaces. We propose a scenario-based trust modeling method in which by varying the degrees of contextual feedback provided to the user, we can measure how the human trust in the system varies under different traffic situations. We then create a ‘trust’ profile for the driver and the autonomous driving behavior can be molded to conform to the trust profile of the user, but only within the confines of overall safety. We show how local interpretability can be used to explain actions of an autonomous car, the operation of which is very complex and hidden from the driver/passenger. For example the AV’s action of stopping suddenly is not enough – the user wants to know why. Without any explanation or context, the user will panic. Our proposed framework can generate such explanations. 
\end{enumerate}


\noindent\emph{\textbf{Broader Impact:}} 
Autonomous control and decision systems are forming the basis for significant pieces of our nation’s critical infrastructure. In particular, autonomous vehicles present direct, and urgent safety-critical challenges. If successful, the research outcomes will have the following impacts: (a) be a valuable contribution towards increasing the overall safety of fully autonomous vehicles, which are likely to become ubiquitous in the near future, (b) the underlying frameworks of generating local explanations from sensor data, and safe operation through reachability analysis can help enhance a large scope of autonomy including but not limited to autonomous vehicles, robotics, aircraft autopilots, and automatic surgery equipment, and (c) Leveraging human trust and emotional behavior to help enhance the capabilities of autonomous vehicles and also facilitate the deployment of autonomous vehicles in the real world. 

\noindent\emph{\textbf{Educational Impact:}} The PI's will develop curriculum including course lectures and hands-on projects related to autonomous driving. The PIs are very vested in promoting and employing undergraduate researchers. They will continue developing and participating in research programs to involve K-12 students into lab research and inspire their interests in autonomy and human factors based upon the hardware and software developed in the proposed research. The PIs will also actively disseminate the research outcomes through outreach in both academia and automotive industries.















