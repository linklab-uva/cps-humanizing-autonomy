%!TEX root = main.tex

%NSF-CPS

Autonomous vehicles (AVs) are not just about getting the technology right, they are about freedom of mobility, and experiences that will literally and figuratively move people in new ways.
The promise of fully autonomous cars has always been attractive, but applying it in a meaningful and coherent way still remains a major challenge. 
Recent accidents involving semi-autonomous cars that resulted in fatalities of either the driver, or in one case the pedestrian, have shaken the confidence and trust of the public in the technology, which does adequately not take the human into account.
%We know now, that it may not always be possible for the driver to take over the control from the autonomous vehicle (AV) at any stage.
Before it’s too late, we need to reevaluate our approach towards autonomy, by seeking answers to questions that put humans at the center stage. 
As journeys become fully automated, the experience itself will need to become more human. 

The goal of the proposed research is to utilize both human, and machine advantages to humanize autonomy by instilling the beneficial nuances of human behavior, emotion, and trust, with the technological and safety benefits of autonomous vehicles.
Behavior-guided autonomous vehicles will bring human factors such as emotions, behaviors, and trust into the autonomous loop, where AVs can enhance the passenger experience, safety, and comfort.
Safety is not primarily just a functional consideration, it is also emotional. We propose that the issues of functional, and emotional design for autonomous vehicles should be tackled together.
%Human behavior and emotions are highly dynamic and are different among individuals based on their previous experiences, environmental factors (e.g., weather, lighting), societal factors, and internal factors (e.g., physiological changes). Currently, AVs (as well as many other autonomous systems) lack in have any sensing and optimization capability according to passenger(s) real-time behavioral and emotional changes. 
Additionally, trust is another core human needs that the AV must establish and defend. However, the nature of trust in the vehicle or the autonomous system varies from individual to individual, and also according to the individual's emotions, and behavior.
%In this research, we will conduct real-life as well as simulation-based experimental studies to identify (1) the association between human emotional attributes and contextual interaction observed from each individual, (2) a taxonomy of emotional and behavioral traits as they relate to the internal and external triggers as well as personalized traits and (3) perspectives of trust in autonomous vehicles from real people, rather than working on assumptions, and (4) the intervention and communication strategies that could be automated by AV to meet passengers(s) need and enhance their "driving/journey" experience.
%Transparency and communication are critical to building trust. 
%To establish user understanding of the system and its capabilities the interface must communicate clearly, and transparently - by revealing what the car sees, what the system is currently doing, what it intends to do in response to environmental conditions and why. 
%For example, emotional down-regulation could be used when passengers might be facing an upsetting or frustrating situation – for instance, a delay in travel. Here the AV could sense the frustration and then down-regulate through voice prompts. When you’re jumping in and out of different AVs, a consistent and personal experience will be vital for successful adoption.
%We’re concerned with a person’s ability, and even right, to make their own decisions, and come and go as they please.
At the end of the day, we are building autonomous vehicles not for robots, but for humans, and therefore, autonomy is not just about how clever cars are without human drivers. But rather, how can autonomous vehicles conform to the human's behavior and emotional needs while still operating under the confines of safety.

\noindent\emph{\textbf{Intellectual Merits:}} The contributions of this proposal are as follows:
\begin{enumerate}[itemsep=0pt,parsep=0pt,topsep=4pt,leftmargin=0.4in]
    \item Modeling driving behavior: We will conduct real-life as well as simulation-based experimental studies to identify the association between human emotional attributes and contextual interaction observed from each individual, develop a taxonomy of emotional and behavioral profiles, and study the perspectives of trust in autonomous vehicles from real people, rather than working on assumptions.
    \item Reachability analysis and control synthesis for safety:
    Based on the behavioral and emotion cues detected in real time, we will dynamically adjust the safety regions for the autonomous vehicle, and generate safe steering and velocity control actions which take driver preferences into account.
    \item Feedback and user interface design: We propose a novel deep learning framework for generating natural language explanations for the AVs actions, based on multi-modal image captioning, and summarization.
    \item We will conduct human driving experiments both using high fidelity autonomous vehicle simulations, a full scale driving simulator, as well as subject-owned vehicles. We will also utilize scaled models and testbed for automotive CPS to evaluate the behavior-guided control synthesis algorithms.
\end{enumerate}


\noindent\emph{\textbf{Broader Impacts:}} 
Autonomous control and decision systems are forming the basis for significant pieces of our nation’s critical infrastructure. In particular, autonomous vehicles present direct, and urgent safety-critical challenges. If successful, the research outcomes will have the following impacts: (a) be a valuable contribution towards increasing the overall safety of fully autonomous vehicles, which are likely to become ubiquitous in the near future, (b) the underlying frameworks of generating local explanations from sensor data, and safe operation through reachability analysis can help enhance a large scope of autonomy including but not limited to autonomous vehicles, robotics, aircraft autopilots, and automatic surgery equipment, and (c) Leveraging human trust and emotional behavior to help enhance the capabilities of autonomous vehicles and also facilitate the deployment of autonomous vehicles in the real world. 

% \noindent\emph{\textbf{Educational Impact:}} The PI's will develop curriculum including course lectures and hands-on projects related to autonomous driving. The PIs are very vested in promoting and employing undergraduate researchers. They will continue developing and participating in research programs to involve K-12 students into lab research and inspire their interests in autonomy and human factors based upon the hardware and software developed in the proposed research. The PIs will also actively disseminate the research outcomes through outreach in both academia and automotive industries.















