The research project will take place at the The Link-Lab - the Cyber-Physical Systems lab located on the University of Virginia’s (UVA) campus. The facilities are partitioned into several laboratories that provide a complete environment for the design, fabrication, and testing of prototype hardware/software systems from initial concept to final implementation. These facilities include sufficient computing and prototyping resources for the proposed research, as described below: 

\begin{description}

\item[A. UVA Link Lab -]

The PI and Co-PIs are members of the Link-Lab at UVA – this space is a new collaborative initiative on Cyber-Physical Systems (CPS) research and education at the University of Virginia. The lab is called “Link Lab” because it “links” multiple engineering departments through cross-cutting mechanisms such as shared lab space, staff, and conference rooms that house faculty and students from multiple disciplines. It houses approximately 20 faculty, 125 students, 3 research scientists, 3 staff, and 6 postdocs. An open floor plan promotes cross-pollination between research groups while at the same time using furniture and layout to provide sound insulation and reduce interruptions.The Link Lab has a large 3000 ft2 open space with moveable tables called the “Arena” that is designed for equipment staging, testbeds, experimental work, as well as a shared common space for seminars, presentations, or social events. It also includes a 2000 ft2 hardware prototyping lab. By locating the space immediately adjacent to the student and faculty desks, and close to the door, this layout will facilitate daily collaboration on this project. Figure 1, illustrates the Link-Lab floor plan. 

\begin{figure*}[h!]
    \centering
    \includegraphics[width = 0.6\linewidth]{figures/linklab.png}
    \caption{Link-lab floor plan}
    \label{fig:floor_plan}
\end{figure*}

\item[B. UVA Viz Lab]
The Viz lab is a facility at UVA designed to help faculty, staff and students explore various 3D visualization tools, such as virtual and augmented reality head mounted displays, for research and education purposes. The staff at Viz lab provide assistance with developing and evaluating virtual and augmented reality environments. 

\item[C. Autonomous Mobile Robots and CPS Lab: ]
Part of the proposed autonomous vehicles testing will be carried out using the testbed available in Co-PI Bezzo’s “Robotic and CPS Lab” which has a large, dedicated, state-of-the-art facility for mobile robotic systems development, prototyping, and control. The lab has been recently renovated and includes support for diverse system development activities with electronics workbenches, flexible space for assembling and experimenting with large demo systems, and secure storage in addition to the computing resources. This space covers an area of more than 900 ft2 with high ceilings and includes: 

\begin{itemize}
    \item Latest generation Vicon Motion Capture System: 8 Vantage cameras (@350 fps) with Lock system and Vicon Tracker software. This system allow sub-millimeter precision localization and tracking of multiple objects moving within the volume of the Lab space
    \item 2 Ascending Technologies Pelican quadrotors with carbon fiber body and equipped with 3rd generation Intel Core i7 CPU, IMU, pressure sensor, GPS, Lidar, stereo cameras, 2.4 GHz XBee link and WiFi - max airspeed 16 m/s, max climb rate 8 m/s, max thrust 36N, max payload 650g.
    \item 1 Ascending Technologies Firefly hexarotor with carbon fiber body and equipped with 3rd generation Intel Core i7 CPU, IMU, pressure sensor, GPS, Lidar, stereo cameras, 2.34GHz XBee link and WiFi (max airspeed 15 m/s, max climb rate 8 m/s, max thrust 36N, max payload 600g.)
    \item 3 Ascending Technologies Hummingbird quadrotors with carbon fiber body and equipped with IMU, pressure sensor, GPS, and 2.4 GHz XBee link - max airspeed 15 m/s, max climb rate 5 m/s, max thrust 20N, max payload 200g.
    \item Several Crazyflie 2.0 nano quadrotors equipped with IMU, pressure sensor, bluetooth, and Qi inductive charger.
    \item Several Parrot Bebop quadrotors equipped with IMU, two cameras, GPS, and sonar.
    \item 2 Clearpath Jackal unmanned ground vehicles, equipped with 3rd generation Intel Core i7 CPU, IMU, NovAtel SMART GPS, Velodyne 3D Lidar, Point Grey Flea3 camera, and WiFi - max speed 2 m/s, max payload 20 kg.
    \item 1 Black-i LandShark military grade 6 wheels UGV equipped with Intel Core i7 CPU, automated turret, Moog Quickset GeminEye, 100x zoom camera, thermal imager, 2 fisheye cameras, 1 camera, 2 Microstrain IMUs, 12 sonar rangers, 12 IR rangers, 2 Hokuyo UTM-30LX Lidars, GPS, and OCU (max speed 10 mph, max payload > 200 lbs)
    \item 1 Stratasys uPrint SE Plus 3D printer
    \item 14 Cisco Systems, standard and high definition dome surveillance cameras with power- over-ethernet capability, along with hardware and software to support surveillance management and leading-edge video analytic completed with high-quality teleconference system and cloud capabilities
\end{itemize}

\item[D. Computational Resources and Rivanna Compute Cluster - ]

The university operates a Linux-based commodity cluster with a frontend named Fir. This cluster is managed by UVa Advanced Computing Services and Engagement and is open to faculty, staff, and graduate students at the University. Undergraduate students and university affiliates are eligible for accounts under faculty sponsorship. Fir is a large-memory cluster consisting of 92 nodes. Twelve of the nodes contain one dual-core 3-GHz Intel dual-core Xeon cpu with 32GB of RAM per node. Another 56 nodes are 8-core servers, with 48 GB per server.  There are also 24 12-way nodes with 96 GB per node.  Most of these cores are hyperthreaded, bringing the total number of logical cores that are eligible for no preemption to 1496. The interconnect for all these nodes is GigE. 

Rivanna, launched in Fall 2014, is the Cray CS300, a 4800-core, high-speed interconnect cluster, with 1.4 PBs of storage. It is composed of 240 compute nodes, each with two 10-core processors and FDR Infiniband interconnect along with a parallel filesystem capable of providing about 25Gb/sec bandwidth. The Cray cluster combines large amounts of processing with large amounts of memory to provide a significant new resource for computationally-intensive research at UVA.

\item[E. Laser Cutters - ]

Campbell Hall has two Universal Laser Systems CO2 lasers. The 50 watt X660M has an 18”x32” bed. The 25 watt M-300 has a 12” x 24” bed capacity. Both can cut virtually any material other than metal, PVC plastics, or anything reflective. These machines can cut and engrave using both vector (lines/shapes) and raster (pixels) modes from virtually any software program.

\item[F. 3D Printer / Rapid Prototyping - ]

The Stratasys Dimension SST 3D printer uses Fuse Deposition Modeling (FDM) technology to build solid ABS plastic model prototypes from 3D stereolithography (stl) files. The machine has an 8”x8”x10” build envelope and builds models in layers down to 0.010 in. thickness.

\item[G. 3-axis Miller and Routing - ]

The MicroMill 2000 and MicroRouter from Denford, Inc. provide full 3-axis CNC and CAM machining capability. Using CAD/CAM software (EdgeCAM / MasterCAM) to generate G-code instructions for the machine, we can translate 2D profiles or 3D solid/surface geometry into machined parts. The router supports 12”x24”x2.5” travel with the ability to feed in and clamp longer stock materials from the side, while the mill supports approximately 9”x3.5”x6” of travel for a single machining operation. Common materials include wood, foam, plastic, aluminum, brass, copper, and mild steel. The machine is also capable of milling marble

\item[H. 3D Digitizer / 3D Laser Scanner - ]

Using technology from MicroScribe and NextEngine, the fabrication facility can both digitize and scan 3D objects and models into CAD systems. The MicroScribe point digitizer captures point, line, spline and surface information using common 3D modeling and CAD software. These can be used to generate surface and solid models of objects, models, topography, and reliefs. The NextEngine 3D Laser Scanner can scan up to a full 360 degree revolution around a small object, creating a full polygon mesh model for export into any 3D CAD software. 

\item[I. Software Resources - ]
The University of Virginia has site licenses for a variety of software for basic computing needs as well as modeling and data analysis, including ANSYS, LabView, Mathcad, and MATLAB. 

\end{description}