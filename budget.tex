\begin{description}

\item[A. Total Budget - ]

\item[B. Personnel - ]

Salary is requested for Dr. M. Behl, PI at 0.5 mos. $(4.17\%)$ CY effort at $\$140,000$ per 12 month appointment, Dr. A. Heydarian, Co-PI at 0.5 mos. $(4.17\%)$$ CY effort at $$\$133,000$ per 12 month appointment, Dr. N. Bezzo, Co-PI at 0.5 mos. $(4.17\%)$ CY effort at $\$128,900$ per 12 month appointment and Dr. L. Feng, Co-PI at 0.5 mos. $(4.17\%)$ CY effort at $\$140,000$ per 12 month appointment.  Because effort as noted in the budget is required for successful completion of the project aims, permission is requested for Drs. Behl, Heydarian, Bezzo and Feng to exceed the two months compensation support per year for all NSF funded projects should it be necessary.  The details of this effort are included in the project description.

\item[C. Graduate Research Assistants (GRAs) and Undergraduate Research Assistants (URAs) -]

Costs are estimated based on the minimum and maximum payments for the academic year established by the University Office of the Vice-President and Provost.  All compensation in SEAS proposals are within these guidelines.  Per UVa policy, GRAs and URAs are limited in the number of hours they can work while taking classes, therefore to calculate hourly rates conversions are made by applying 1056 (GRA) and 840 (URA) hours per calendar year.  The support provided for GRAs also includes tuition and health insurance shown below as Other costs.  Salary is requested for 3 fulltime GRAs for 12 calendar months.

\item[D. Salary Increases - ]

Salary increases of $3\%$ per year (calculated effective July 1st) are from the University's Multi‐Year Financial Plan used by the Board of Visitors and administration to guide the University of Virginia in long‐term financial planning. The plan is also submitted to the State of Virginia. The projected rate for salary increases is based on available competitive salary surveys with other institutions. 

\item[E. Fringe Benefits - ]

The University of Virginia's fringe benefits rates as they apply to sponsored programs are as follows:  $28.2\%$ for faculty, research staff, postdoctoral fellows and professional staff, $39.10\%$ for classified staff, $28.2\%$ for part-time faculty and staff; $5.7\%$ for temporary employees and wage employees.  Fringe Benefits can include:  FICA/Medicare, Retirement, Disability Insurance, Life Insurance, TIAA/CREF, Workers’ Compensation, Unemployment Insurance and Health Insurance.

\item[F. Travel - ]

Trips to related technical conferences, workshops, seminars, etc.  Trips to sponsor for technical discussions and presentation of results. Travel funds are budgeted primarily to travel to Cyber-Physical Systems (CPS) Week, International Conference of Machine Learning (ICML) and the International Conference on Robotics and Automation  (ICRA) conference each year. 
\arsalan{Add a conference venue for human factors}

\item[G. Materials and Supplies - ]

Laboratory supplies for specific use in the research project are requested. These include : \textbf{[describe equipment]}

\item[H. Departmental Computer Facilities Fees -]

Departmental Computer Service Fees: the Computer Science department operates an extensive computing facility in support of its research activities.  120 powerful, multicore compute servers, with 792 total cores (and some nodes enabled with high-end GPUs) are available through infrastructure-as-a-service and software-as-a-service interfaces.  Research software (e.g., R, SAS, SPSS, MATLAB, Simics) is available for interactive use on research workstations and for high-throughput use on the server cluster.  Research storage provides 180 TB of RAID total storage with multiple backup modes and failover.  Departmental computing resources are interconnected by switched gigabit Ethernet, with Infiniband connectivity within clusters.  The facility also provides staff to help support specialized programming, specialized database and web services, use of local and remote high-performance computing facilities, and setup/maintenance for custom hardware/OS/network/software infrastructure for sponsored research.  Rates are established annually by the managing unit and approved by the University Office of the Comptroller in compliance with OMB rules governing cost-recovery centers.  Rate calculations consist of personnel, equipment depreciation, supplies and materials, and profit/(loss).  The current computing service rate is $\$3.16$ per hour.  Rates are applied equally to all users of the Research Facility.

Other:
\begin{enumerate}
    \item Tuition Remission - Effective September 1, 1990, it is the policy of the University of Virginia to provide tuition for graduate research assistants as partial compensation for services.  Year 1: $\$34,380$, Year 2: $\$35,583$, Year 3; $\$36,829$.
    \item b.	Graduate Research Assistant Health Insurance – Effective July 1, 2005, it is the policy of the University of Virginia to provide health insurance for graduate research assistants as partial compensation for services.  Health insurance is increased $5\%$ each year to cover future rate increases. Year 1: $\$5,660$, Year 2: $\$5,943$, Year 3: $\$6,240$
\end{enumerate}

\item[I. Facilities and Administrative (F\&A) (Indirect/Overhead) Costs - ]
Facilities and Administrative (F\&A) (Indirect/Overhead) Costs - The University of Virginia's negotiated (Modified Total Direct Costs (MTDC) F\&A rates with DHHS, per agreement of June 27, 2017, is $61\%$ "on campus" and $26\%$ "off-campus" until 6/30/18 and $61.5\%$ starting 7/1/18.  The “off-campus” rate remains the same.  (Note: The MTDC base consists of total direct costs less individual equipment items in excess of $\$5,000$, alterations and renovations, patient care costs, tuition remission and rental costs of off-campus facilities.)  Includes F\&A on the first $\$25,000$ of subcontracts.

\end{description}