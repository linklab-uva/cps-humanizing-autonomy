\subsection{Research Thrust 3: Feedback Design}
\label{sec:trust}

The purpose of this research thrust is to develop and validate models to quantify trust of a driver in an autonomous vehicles.
Trust in self-driving cars is one of the big discussion points in the public debate. 
Drivers who have always been in complete control of their car are expected to willingly hand over control and blindly trust a technology that could kill them.
We hypothesize that trust is influenced by three components:
\begin{enumerate}[itemsep=0pt,parsep=0pt,topsep=4pt,leftmargin=0.4in]
    \item The person who trusts,
    \item The system this person is supposed to trust, and
    \item The driving situation.
\end{enumerate}

In addition, the first component (i.e., the person), is characterized by a certain propensity to trust, which is influenced by different factors (e.g., gender, age, opinions, character traits). This is what we want to measure in the proposed research. 

While cars have become significantly more usable — particularly with regard to reliability and safety over the past twenty years — thanks to the introduction of new technologies such as electronic fuel injection, the seat belt, crumple zones, ABS, airbags, electronic stability control and GPS satellite navigation, many of these technologies have succeeded out-of-sight of the humans behind the wheel.
Yet when it comes to newer technologies - both on-board telematics,communication and the ADAS, we see a much less successful integration of technology, vehicle and user.
At the broadest level, many of the technologies available in modern cars do not appear to have been developed with a particular user-centred approach. 
They exist because the technology has become available to perform a specific function.

As soon as driving ``feels'' even partly autonomous, people switch off, they become disengaged from the process of driving — and fail to monitor the system. 
A quick YouTube search, reveals many videos where people are aware the systems have limitations, but still push them further than their intended use, operating pilot assist systems on roads or situations when they shouldn't.

Trust comes from two factors: predictability and explainability.
If a user expects a car to drive in a certain way in a certain situation, and the car conforms to his expectation, the user will tend to trust it more.
On occasion, when the AV’s action surprise/confuse the user- as long as there is an explanation provided for it, the user can again gauge her level of trust int he system.
Our goal is :
\begin{enumerate}[itemsep=0pt,parsep=0pt,topsep=4pt,leftmargin=0.4in]
    \item Understand the set of trust expectations a driver has for their autonomous vehicle.
    \item Develop a set of explanations autonomous vehicles should provide to promote trust and create an algorithm that can provide these explanations.
    \item Perform experiments to understand how content, timing and delivery impacts the effectiveness of explanations.
\end{enumerate}

\subsubsection{Scenario-based trust modeling}
\label{subsec:trust-modeling}

\madhur{Madhur describes the scenario based experiments with prescan}
% feel free to create another tex input file for this subsection 
\madhur{I will add some text to this subsection as well.}


\subsubsection{Deep explanations}
\label{subsec:explainability}
\madhur{madhur to add.}

\begin{figure}
    \centering
    \includegraphics[width=\columnwidth]{figures/deep_exp.pdf}
    \caption{Deep-Explanation generation: Each dimension of the scene decomposition is used as an input to caption generation. Representation matching, and seq2seq are then used to generate a likely explanation for the predominant action stream.  }
    \label{fig:deep_exp}
\end{figure}

Automatic image description generation is a challenging problem that has recently received a large amount of interest from the computer vision and natural language processing communities ~\cite{johnson2016densecap, xu2015show, wang2016image, karpathy2015deep,Vinyals2015ShowAT}
Not only must caption generation models be able to solve the computer vision challenges of determining what objects are in an image, but they must also be powerful enough to capture and express their relationships in natural language. 
For this reason, caption generation has long been seen as a difficult problem.
The task of automatic image description involves taking an image, analyzing its visual content, and generating a textual description (typically a sentence) that verbalizes the most salient aspects of the image. 
This requires the joint use of both Computer Vision and Natural Language Processing techniques.
Yet despite the difficult nature of this task, there has been a recent surge of research interest in attacking the image caption generation problem. In particular, deep neural networks have been shown to form new grammatically correct sentences as opposed to the template based models and their limited generalization capability to a novel image.
To capture the correlation between two modalities i.e. visual and natural language we need to map both these to some same space so at learn the relation between them or say we need to learn the multimodal joint model.
Models that uses different deep neural networks like convolutional neural network (CNN), long short term memory(LSTM) networks, recurrent neural network(RNN) to implicitly learn the common embedding. These by far gives the best result on all common datasets of caption generation
Aided by advances in training deep neural networks and the availability of large classification datasets, recent work has significantly improved the quality of caption generation using a combination of convolutional neural networks (convnets) to obtain vector representation of images and recurrent neural networks to decode those representations into natural language sentences.

In the proposed research we extend attention-based image caption generators to work with multidimensional data-sets.
\begin{enumerate}
    \item Instead of generating captions from RGB images alone, we will also generate captions from LIDAR data, depth sensor images, and segmented images. 
    \item The captions themselves, will be enhanced with information about the control decision (steering, acceleration, and braking) made.
    \item We will gather and release a multidimensional caption data-set specifically for autonomous vehicles. 
\end{enumerate}


