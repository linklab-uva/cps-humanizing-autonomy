\subsection{Research Thrust 1: Behavior modeling}
\label{sec:behaviour}

One of the most compelling benefits of emotion-aware vehicles is the ability to monitor drivers’ behavior and address potential safety concerns associated with facial expressions and mood.
Identifying fatigue, distraction, and frustration to prevent accidents before they happen.
If a self-driving car perceived emotional distress from passengers, it could drive more slowly or play soothing music to assuage their anxiety.
Autonomous driving machines adjust driving styles based on the occupants’ non-verbal feedback.

he proposed research aims to bring human factors such as emotions, behaviors, and trust into the autonomous loop, where AVs can enhance the passenger(s) experience, safety, and comfort. Human behavior and emotions are highly dynamic and are different among individuals based on their previous experiences, environmental factors (e.g., weather, lighting), societal factors, and internal factors (e.g., physiological changes). Currently, AVs (as well as many other autonomous systems) lack in have any sensing and optimization capability according to passenger(s) real-time behavioral and emotional changes. Additionally, trust is another core human needs that the AV must establish and defend. However, the nature of trust in the vehicle or the autonomous system varies from individual to individual as well. In this research, we will conduct real-life as well as simulation-based experimental studies to identify understand (1) t the association between human emotionalhe causation of attributes and contextual interaction observed from each individual, (2) a taxonomy of emotional and behavioral traits as they relate to the internal and external triggers as well as personalized traits and (3) perspectives of trust in autonomous vehicles from real people, rather than working on assumptions, and (4) the intervention and communication strategies that could be automated by AV to meet passengers(s) need and enhance their "driving/journey" experience.. 



Safety is not primarily just a functional consideration, it is also emotional. We propose that the issues of functional and emotional design for autonomous vehicles should be tackled together.
For example, emotional down-regulation could be used when passengers might be facing an upsetting or frustrating situation – for instance, a delay in travel. Here the AV could sense the frustration and then down-regulate through voice prompts. When you’re jumping in and out of different AVs, a consistent and personal experience will be vital for successful adoption.
 We’re concerned with a person’s ability, and even right, to make their own decisions, and come and go as they please. Not about how clever cars are without human drivers. 




Talk about 
Specifically, we are interested in formally characterizing emotions through a “context-dependence” approach, where spatial and temporal information can be automatically detected, analyzed, and interpreted. We are also interested in developing a database of  emotions, behaviors and environmental changes as there exists a limited number of emotional models and databases available to the public and the research community.



In this research, we are investigating the influence of environmental factors on emotional traits and consequently passenger behaviors. Through instrumentation of a number of different manual and semi-automatic vehicles, we will collect environmental data with ambient-condition (i.e., thermal conditions, noise levels, and available lighting) sensors in parallel with user-specific behavioral, emotional and physiological traits by using eye-tracking devices, wearables (smartwatch), chair sensors, along with facial recognition algorithm; additionally, music social media (e.g., Spotify, Pandora) monitoring will be used to identify changes in participants’ emotions that may lead to identification of certain behaviors and responses. Through statistical analysis and machine learning techniques, we will identify models of behavioral and emotional traits and explore the development of behavior-emotion-interventions taxonomy.  specific research questions include: 
1.	How to detect behavior and emotions non-intrusively and with least number of sensors?
2.	What is the relationship between environmental changes and passengers emotions and behaviors?
3.	What is the taxonomy of behaviors and emotions in driving?






Autonomous cars are not just about the technology. They are about freedom of mobility, and a whole set of experiences that will literally and figuratively move people in new ways.



While the promise of self-driving cars is attractive, applying it in a meaningful and coherent way, where passenger(s) behaviors, preferences, and needs are considered, still remains a limitation and major challenge.


Transparency and communication are critical to building trust. To establish user understanding of the system and its capabilities the interface must communicate clearly, and transparently - by revealing what the car sees, what the system is currently doing, what it intends to do in response to environmental conditions and why. 




\arsalan{Spill the magic dust Arsalan}