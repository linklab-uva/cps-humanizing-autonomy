\subsection{Research Thrust 1: Behavior modeling}
\label{sec:behaviour}

 The ability to (1) detect, (2) assess and (3) control a person’s emotions has been identified to be the predictor of success in relating to the people around us \AH{add a sentence here that if cars want to build trust with humans just like how we can relate ourselves to other humans then cars need to know how we feel and what are emotions are like}. By being able to read other’s “emotional cues,” not only we can better understand how they are feeling at a given time, but it also helps us to predict how they will respond in different scenarios. Research suggests that emotions are normally associated with specific events or occurrences, and they can significantly influence our thoughts and behaviors. Additionally, we can use reason to evaluate our emotions, interpret them, and reassess our initial reactions to them. Therefore, by detecting certain events and occurrences, we may be able to assess and predict individual emotional states and use reason to soften their impact or “shift” their meaning. 



We aim to better understand what environmental factors influence passenger emotions and how these emotions later influence behaviors.  After having a better understanding of these factors, we aim to develop psychological interventions to (1) reduce the negative outcomes of experiencing particular emotions \AH{this needs to be re framed or updated changed} and (2) reduce the likelihood of passengers’ experiencing triggers that lead to emotions that have negative consequences (e.g., safety, trust). 
As a first approach, we plan to observe trends in our data that both reveal (1) factors that influence specific human emotions and (2) the downstream consequences of particular emotions on their behaviors. For example, we will aim to understand what environmental (e.g., weather, thermal conditions, lighting, noise) and social factors (e.g., social interaction), physiological factors (e.g.,arm movement) trigger particular emotions. We will also aim to understand how particular emotions influence safety, attention to surrounding environment, social interactions and.... 

Specifically, we are interested in formally characterizing emotions through a “context-dependence” approach, where spatial and temporal information can be automatically detected, analyzed, and interpreted. We are also interested in developing a database of  emotions, behaviors and environmental changes as there exists a limited number of emotional models and databases available to the public and the research community.

As it may be difficult to often identify triggers to particular emotions (i.e., passenger(s) may appear already experiencing a particular emotion), we plan to evaluate which behavioral outcomes can be used as cues for particular emotions. For example, we may learn that when participants are feeling sad, they are more likely to deviate from their normal accelerating and decelerating behaviors or more likely to listen to certain genre of music. With this information, we will then assess if we can reliably predict particular emotions, given that a passenger behaved in a particular way. With this information as a cue, we can understand that there might be a need to intervene or provide some feedback to the passenger at that time. 


In this research, we are investigating the influence of environmental factors on emotional traits and consequently passenger behaviors. Through instrumentation of a number of different manual and semi-automatic vehicles, we will collect environmental data with ambient-condition (i.e., thermal conditions, noise levels, and available lighting) sensors in parallel with user-specific behavioral, emotional and physiological traits by using eye-tracking devices, wearables (smartwatch), chair sensors, along with facial recognition algorithm; additionally, music social media (e.g., Spotify, Pandora) monitoring will be used to identify changes in participants’ emotions that may lead to identification of certain behaviors and responses. Through statistical analysis and machine learning techniques, we will identify models of behavioral and emotional traits and explore the development of behavior-emotion-interventions taxonomy.  specific research questions include: 
\begin{enumerate}
    \item How to detect behavior and emotions non-intrusively and with least number of sensors?
    \item What is the relationship between environmental changes and passengers emotions and behaviors?
    \item What is the taxonomy of behaviors and emotions in driving?
    \item Whether certain behaviors can indicate trust? for instance, do you trust your own behavior the most? 
    \item What are environmental and social factors that can gain your attention? (research on signs)
\end{enumerate}



\AH{add about how cars can enhance the driver emotions as well but selecting routes or choosing music or behaving in a certain way...}







%One of the most compelling benefits of emotion-aware vehicles is the ability to monitor drivers’ behavior and address potential safety concerns associated with facial expressions and mood.
%Identifying fatigue, distraction, and frustration to prevent accidents before they happen.
%If a self-driving car perceived emotional distress from passengers, it could drive more slowly or play soothing %music to assuage their anxiety.
%Autonomous driving machines adjust driving styles based on the passenger(s) non-verbal feedback.

%he proposed research aims to bring human factors such as emotions, behaviors, and trust into the autonomous loop, where AVs can enhance the passenger(s) experience, safety, and comfort. Human behavior and emotions are highly dynamic and are different among individuals based on their previous experiences, environmental factors (e.g., weather, lighting), societal factors, and internal factors (e.g., physiological changes). Currently, AVs (as well as many other autonomous systems) lack in have any sensing and optimization capability according to passenger(s) real-time behavioral and emotional changes. Additionally, trust is another core human needs that the AV must establish and defend. However, the nature of trust in the vehicle or the autonomous system varies from individual to individual as well. In this research, we will conduct real-life as well as simulation-based experimental studies to identify understand (1) the association between human emotional he causation of attributes and contextual interaction observed from each individual, (2) a taxonomy of emotional and behavioral traits as they relate to the internal and external triggers as well as personalized traits and (3) perspectives of trust in autonomous vehicles from real people, rather than working on assumptions, and (4) the intervention and communication strategies that could be automated by AV to meet passengers(s) need and enhance their "driving/journey" experience.. 



%Safety is not primarily just a functional consideration, it is also emotional. We propose that the issues of functional and emotional design for autonomous vehicles should be tackled together.
%For example, emotional down-regulation could be used when passengers might be facing an upsetting or frustrating situation – for instance, a delay in travel. Here the AV could sense the frustration and then down-regulate through voice prompts. When you’re jumping in and out of different AVs, a consistent and personal experience will be vital for successful adoption.
% We’re concerned with a person’s ability, and even right, to make their own decisions, and come and go as they please. Not about how clever cars are without human drivers. 




%Autonomous cars are not just about the technology. They are about freedom of mobility, and a whole set of experiences that will literally and figuratively move people in new ways.



%While the promise of self-driving cars is attractive, applying it in a meaningful and coherent way, where passenger(s) behaviors, preferences, and needs are considered, still remains a limitation and major challenge.


%Transparency and communication are critical to building trust. To establish user understanding of the system and its capabilities the interface must communicate clearly, and transparently - by revealing what the car sees, what the system is currently doing, what it intends to do in response to environmental conditions and why. 




\arsalan{Spill the magic dust Arsalan}