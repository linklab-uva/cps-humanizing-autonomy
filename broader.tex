%!TEX root = main.tex


\section{Broader Impacts}
\label{sec:broader}

%Autonomous control and decision systems are forming the basis for significant pieces of our nation’s critical infrastructure. In particular, autonomous vehicles present direct, and urgent safety-critical challenges. 
As the development of AVs progresses at faster rates than ever utilizing a diverse set of approaches, it is essential that frameworks for human centered design be developed in close coordination with academic and industry partners. 
We are working closely with automotive manufacturers like Toyota, as well as Perrone Robotics - an autonomous driving company based in Virginia. Letters of collaboration form both are included in the grant application.
The research outcomes will have the following broader impacts: (a) be a valuable contribution towards increasing the overall safety of fully autonomous vehicles, which are likely to become ubiquitous in the near future, (b) minimize the risk of economic loss, damages, and injuries due to failures and uncertainties, (c) deep explanations from sensor data can help enhance a large scope of autonomy including but not limited to autonomous vehicles, robotics, aircraft autopilots, and automatic surgery equipment, and (d) Leveraging human behavior, emotions, and trust to help enhance the capabilities of autonomous vehicles and also facilitate the deployment of autonomous vehicles in the real world.
% !TEX root = propgen.tex
%\vspace{-5pt}
%\subsection{NICOLA Broader Impacts}
%\vspace{-5pt}
% \noindent {\bf Autonomous Robots Curriculum Development:}
% PI Bezzo is collaborating with several engineering faculty at UVA to develop a robotics curriculum that would serve advanced undergraduate and graduate students. In 2016 Bezzo and other three colleagues won an educational innovation award from the School of Engineering in support of curriculum development, and in the 2016-17 academic year the group offered 5 new robotics courses.   Four technical courses, }; {\em Cooperating Autonomous Robotics}; {\em Sensors \& Perception} were augmented by a seminar {\em Robots \& Society}.  
% The Electrical and Computer Engineering and Systems Engineering programs contributed funds that enabled the purchase of 10 turtlebots, 10 crazyflie quadrotors, and 12 NAO robots, used in these courses for lab exercises. The work in this proposal will further contribute to the robotics curriculum by including lectures in the current course content, running seminars on the topic of trusted and assured autonomy, machine learning in robotics, scheduling and replanning, and by establishing a new advanced course on {\em Assured Autonomy}. 
%that welcomes students from several programs (EE, CS, CpE, SIE, MAE) and complements work in the Link Lab -- 
%PI Bezzo teaches a graduate course on Autonomous Mobile Robots and an undergraduate course on Robotics and CPS Simulations. 
%Additionally,  the proposed research will be conducted as part of the Link Lab --
% a major initiative in CPS recently launched at UVA. 
%UVA has launched undergraduate and masters-level educational initiatives in cyber-physical systems. The proposed research will provide valuable case studies and laboratory experiences in support of these initiatives. 
% PI Bezzo teaches also an undergraduate course on CPS Simulations. The proposed research and results will be included in the curriculum activities for this course to include safety assessments during simulations for autonomous vehicles \NB{may need to elaborate this}
% % PI Bezzo has also raised \$15k funds from UVA Dean's Office to create a shared laboratory space for education activities on resilience in CPS: the Lab will consist of hardware in the loop simulators of ground and aerial mobile robots accessible first over the UVA cloud network and then externally, similar to the concept introduced by the Robotarium at Georgia Tech \cite{robotarium} but focused on activities about CPS resiliency and online adaptation and safety assesment, as presented in this work.
% The PI has been leading also several outreach activities like live demonstrations at local high schools, Open Lab days, and summer camps which will be continued through the duration of this project to expose students to the problems presented in this proposal. Through the use of the testbed of turtlebots, PI Bezzo plans to run robotics summer camps for local middle and high schoolers to enagage them to the problem of safety in autonomous systems.
% PI Bezzo strongly believes that  these education and outreach activities will be a catalyst to attract more students to pursue a career in STEM. 
% Furthermore, research may benefit from all of these education activities which can be viewed as crowdsourcing to reveal new issues and ways to deal with the problem of safe navigation of autonomous robotic systems.
%simply by leveraging crowdsourcing which may reveal new issues and ways to deal with the problem of safe navigation of autonomous robotic systems.
%online scheduling and replanning of autonomous system operations. 
%Furthermore, exposing students to the problems proposed in this work may lead to crowdsourcing activities and reveal new issues and ways to deal with online scheduling and replanning of autonomous system operations.  
%initiatives
% \noindent {\bf Broadening Participation:} The PI has already gained experience in involving undergraduates in research, including advising several minority students through Research Experience for Undergraduates awards. Currently PI Bezzo has six students in his group including a woman pursuing a PhD and an African American woman pursuing an undergraduate degree in computer engineering with concentration in robotics. PI Bezzo will involve both undergraduate and graduate students into the project, with emphasis on recruiting underrepresented groups through collaboration with UVA's Center for Diversity in Engineering and programs such as the NSF REU program for African American students, UVA chapters of the Society of Women Engineers, National Society of Black Engineers, and Society of Hispanic Professional Engineers.  
% UVA holds an Engineering Open House, which provides an  introduction to engineering for K-12 students, along with a number of specialized programs such as Cyber Camp. Since joining UVA, the PI has engaged in these activities using demonstrations with his testbed of mobile robots which he plans to continue and enlarge to cover the problem of online reachability and adaptation presented in this project.
% \noindent {\bf Catalyst for Parallel and Future Research and Technology Transfer:}
% While the proposed work is set in the context of autonomous vehicles, it is applicable to CPS broadly and will contribute directly to the development of safe autonomous systems, increase and improve safety, and minimize the risk of economic loss, damages, and injuries due to failures and uncertainties. % Finally, all research results will be presented at major conferences and published in refereed journals while codes will be posted at {\tt{github.com}} and {\tt{ros.org}}.


\subsection{Improving Education on Autonomy and Cyber-Physical Systems:}
%There is a significant gap in the way we conduct interdisciplinary Cyber-Physical Systems (CPS) research, and the way we train students about CPS. Students coming out of higher education are expected to solve 21st-century CPS problems and enter into occupations that haven’t even been imagined yet. 
%The PIs teaching mirrors the inter-disciplinary approach towards their research. 
The PIs will develop new courses to ensure that students cultivate a holistic view of life-critical, and safety-critical system development by drawing stronger connections between systems theory, formal methods, machine learning, human factors, and hands-on development.
%New courses for the graduate and undergraduate teaching will be devleoped:
\begin{enumerate}[itemsep=0pt,parsep=0pt,topsep=4pt,leftmargin=0.4in]
    \item \textbf{Principles of Modeling in Cyber-Physical Systems} : This course will provide a solid foundation for understanding different modeling paradigms, and explore them through a deep dive and hands on implementation for three CPS domains: Energy, Medical, and Automotive cyber-physical systems. Students will come out of this course with advanced and transferable knowledge of model-based design methods and tools, and will be ready for tackling multi-disciplinary systems projects. 
    \item \textbf{F1/10 Autonomous Racing} - PI Behl teaches a course where teams of students build, drive, and race a fleet of 1/10 scale fully autonomous vehicles, while learning about principles of perception, planning, and control. The F1/10 platform facilitates a wide range of research, education, and training in autonomy. The course material developed by PI Behl for F1/10 is open-source, and publicly available on f1tenth.org. It has been used by a dozen universities around the world to build their own versions of the 1/10 autonomous cars. His course materials have been adapted for teaching CPS and Autonomous Systems courses at UT Austin, and Clemson University.
    \item The Co-PIs Bezzo and Heydarian, will also co-teach a new course on \textbf{Robots \& Humans} which explicitly focuses on the interplay of trust and behavior with autonomous vehicles.
    \item Co-PI Bezzo also teches an undergraduate course on \textbf{CPS Simulations}. The proposed research and results will be included in the curriculum activities for this course and will utilize C0-PI Feng's full scale driving simulator.
\end{enumerate}
All the PIs reside within the Cyber-Physical Systems Link Lab at UVA and will jointly work together towards creating a template for CPS education centered around behavior guided autonomy, and autonomous vehicles. 
\newline
\noindent[\textbf{K-12 Impacts:}]
The PIs have organized several outreach programs and workshops (Iridescent, C-TECH2 Workshop, NASA INSPIRE Workshop) promoting STEM studies to underrepresented and minority students in K-12. 
We plan to run robotics summer camps for local middle and high schools to engage them with safety in autonomous vehicles.
PI Heydarian collaborates with the ``Girls Can Change the World with Science'' outreach program to inspire young female students across Virginia elementary and middle schools to consider futures in science and engineering fields. 
%We use the findings of this research to demonstrate how some of our engineering-problems can only be solved by integrating engineering fields (e.g., computer science, system engineering) and other domain(s) together (i.e., psychology). 

\newline
\noindent[\textbf{Graduate/Undergraduate Students and Outreach Effort:}]
The PIs are committed to recruiting and nurturing minority students by actively participating in local programs such as Women in Computer Science (WICS),  Open-house visitation days, and 1-on-1 career-related advising.
Every year, for the past 3 years, PI Behl has been organizing the International F1/10 Autonomous Racing Competition held in conjunction with a premier venue such as CPS Week (2018), SenSys (2017), and ES-Week (2016, and 2018 Fall).
%, where teams from all over the build 1/10 scale cars using open source instructions on how to build, drive, and race these vehicles in a battle of algorithms. The competitions are  
%In additional to the organizing the competition, the PI has regularly held F1/10 tutorials to teach undergraduates, and graduate students about autonomy. 
PI Feng organized an N$^2$Women Luncheon at the CPS Week 2015, and is co-organizing a Mentoring Workshop at FLoC 2018. 
%The 4 PIs collectively advise 7 PhD students and have mentored over two dozen undergraduate students.
We will also engage in outreach activities aimed to recruit graduate students from neighboring Historically Black Colleges and Universities, particularly Virginia State University.

\newline
\noindent[\textbf{Dissemination Impacts}]
% The findings, results, and datasets generated from the proposed testbed will become available to researchers, where they can test different methods and form scientific comparisons of different approaches. We will release general findings, developed software interfaces, algorithms and models via the Web so the interested community (academia and industry) can use and implement our tools and techniques in their work. The scientific community will be made aware of the research project and opportunities for collaboration through the project and Link-Lab webpages, seminars and conferences such as xxxx and SenSys, and mailing lists of targeted communities such as the Society for Personality and Social Psychology, xxxxxxx %add whatever else that is needed
The scientific community will be made aware of the research project and opportunities for collaboration through the project and Link-Lab webpages, seminars and conferences such as CPS Week, IROS, RAS, ICML, HCI, and SenSys
%PI Behl will maintain a website dedicated to the research focus and publish all results, presentations, and videos there, consistent with the Data Management Plan. 
The PIs are regularly invited by Toyota, US DoT, Virginia Department of Transportation (VDoT), to give invited talks to the AV community.
%PI Behl serves on the editorial board of the SAE International Journal of Connected and Automated Vehicles.
This research will lead to the public release of a number of models, data-sets, and tools:
\begin{enumerate*}
    \item Model and data-set library for driving behaviors and emotions. 
    \item Annotated training data for learning natural language explanations from multimodal sensor data.
    \item A UI toolkit with Unity elements for designing feedback for autonomous vehicles
    \item Github repository for code and tutorials related to deep explanations, and control synthesis algorithms. 
\end{enumerate*}
%The PIs are fully committed to these education and outreach activities and believe these will be a catalyst to attract more students to pursue a career in STEM.