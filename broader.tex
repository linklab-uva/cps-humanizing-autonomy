\section{Broader Impacts}
\label{sec:broader}

Autonomous control and decision systems are forming the basis for significant pieces of our nation’s critical infrastructure. In particular, autonomous vehicles present direct, and urgent safety-critical challenges. The research outcomes will have the following broader research impacts: (a) be a valuable contribution towards increasing the overall safety of fully autonomous vehicles, which are likely to become ubiquitous in the near future, (b) the underlying frameworks of generating local explanations from sensor data, and safe operation through reachability analysis can help enhance a large scope of autonomy including but not limited to autonomous vehicles, robotics, aircraft autopilots, and automatic surgery equipment, 
(c) provide valuable and scientific insights to automotive manufactures and stakeholders about user interface design for sled-driving cars, and user expectations about fully autonomous cars, and (d) Leveraging human behavior, emotions, and trust to help enhance the capabilities of autonomous vehicles and also facilitate the deployment of autonomous vehicles in the real world.

\subsection{Improving Education on Autonomy and Cyber-Physical Systems:}

There is a significant gap in the way we conduct interdisciplinary Cyber-Physical Systems (CPS) research, and the way we train students about CPS. Students coming out of higher education are expected to solve 21st-century CPS problems and enter into occupations that haven’t even been imagined yet. The PIs teaching mirrors the inter-disciplinary approach towards research. 
The PIs will develop new courses to ensure that students cultivate a holistic view of life-critical, and safety-critical system development by drawing stronger connections between systems theory, formal methods, machine learning, human factors, and hands-on development.
PI Behl has pioneering two new courses for the graduate and undergraduate teaching:
\begin{enumerate}
    \item \textbf{Principles of Modeling in Cyber-Physical Systems} : This course provides a solid foundation for understanding different modeling paradigms, and explore them through a deep dive and hands on implementation for three CPS domains: Energy, Medical, and Automotive cyber-physical systems. Students come out of this course with advanced and transferrable knowledge of model-based design methods and tools, and will be ready for tackling multi-disciplinary systems projects. All the lectures are available online for anyone to view. 
    \item \textbf{F1/10 Autonomous Racing} - Principles of Perception, Planning, and Control. - Teams of students build, drive, and race a fleet of 1/10 scale fully autonomous vehicles, while learning about principles of perception, planning, and control. The F1/10 platform facilitates a wide range of research, education, and training in autonomy. The course material developed by PI Behl for F1/10 is free and open-source, and publicly available on f1tenth.org. It has been used by a dozen university around the world to build their own versions of the 1/10 autonomous cars. PI Behl's course materials available online have been adapted for teaching CPS and Autonomous Systems courses at UT Austin, and Clemson University.
\end{enumerate}

All the PIs reside within the Cyber-Physical Systems Link Lab at UVA and will jointly work together towards creating a template for CPS education centered around behavior guided autonomy, and autonomous vehicles. 

\subsection{K-12 outreach Impacts:}

\subsection{Graduate/Undergraduate Students and Outreach Effort:}
The PIs are committed to recruiting and nurturing minority and local high-school students by actively participating in local programs such as Women in Computer Science (WICS),  Open-house visitation days, and 1-on-1 career-related advising.
Every year, for the past 3 years, PI Behl has been organizing the International F1/10 Autonomous Racing Competition, where teams from all over the build 1/10 scale cars using open source instructions on how to build, drive, and race these vehicles in a battle of algorithms. The competitions are held in conjunction with a premier venue such as CPS Week (2018), SenSys (2017), and ES-Week (2016, and 2018 Fall). In additional to the organizing the competition, the PI has regularly held F1/10 tutorials to teach undergraduates, and graduate students about autonomy.  

\subsection{Dissemination Impacts:}
PI Behl will maintain a website dedicated to the research focus and publish all results, presentations, and videos there, consistent with the Data Management Plan. 
In addition, the PIs are regularly invited by Toyota, US DoT, Virginia Department of Transportation (VDoT), to give invited talks to the AV community.
PI Behl serves on the editorial board of the SAE International Journal of Connected and Automated Vehicles.