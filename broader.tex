\section{Broader Impacts}
\label{sec:broader}

Autonomous control and decision systems are forming the basis for significant pieces of our nation’s critical infrastructure. In particular, autonomous vehicles present direct, and urgent safety-critical challenges. 
As the development of AVs progresses at faster rates than ever utilizing a diverse set of approaches, it is essential that frameworks for human centered design be developed in close coordination with academic and industry partners. 

The research outcomes will have the following broader research impacts: (a) be a valuable contribution towards increasing the overall safety of fully autonomous vehicles, which are likely to become ubiquitous in the near future, (b) the underlying frameworks of generating local explanations from sensor data, and safe operation through reachability analysis can help enhance a large scope of autonomy including but not limited to autonomous vehicles, robotics, aircraft autopilots, and automatic surgery equipment, 
(c) provide valuable and scientific insights to automotive manufactures and stakeholders about user interface design for sled-driving cars, and user expectations about fully autonomous cars, and (d) Leveraging human behavior, emotions, and trust to help enhance the capabilities of autonomous vehicles and also facilitate the deployment of autonomous vehicles in the real world.

\subsection{Improving Education on Autonomy and Cyber-Physical Systems:}

There is a significant gap in the way we conduct interdisciplinary Cyber-Physical Systems (CPS) research, and the way we train students about CPS. Students coming out of higher education are expected to solve 21st-century CPS problems and enter into occupations that haven’t even been imagined yet. The PIs teaching mirrors the inter-disciplinary approach towards research. 
The PIs will develop new courses to ensure that students cultivate a holistic view of life-critical, and safety-critical system development by drawing stronger connections between systems theory, formal methods, machine learning, human factors, and hands-on development.
PI Behl has pioneering two new courses for the graduate and undergraduate teaching:
\begin{enumerate}[itemsep=0pt,parsep=0pt,topsep=4pt,leftmargin=0.4in]
    \item \textbf{Principles of Modeling in Cyber-Physical Systems} : This course provides a solid foundation for understanding different modeling paradigms, and explore them through a deep dive and hands on implementation for three CPS domains: Energy, Medical, and Automotive cyber-physical systems. Students come out of this course with advanced and transferable knowledge of model-based design methods and tools, and will be ready for tackling multi-disciplinary systems projects. All the lectures are available online for anyone to view. 
    \item \textbf{F1/10 Autonomous Racing} - Principles of Perception, Planning, and Control. - PI Behl teaches a course where teams of students build, drive, and race a fleet of 1/10 scale fully autonomous vehicles, while learning about principles of perception, planning, and control. The F1/10 platform facilitates a wide range of research, education, and training in autonomy. The course material developed by PI Behl for F1/10 is free and open-source, and publicly available on f1tenth.org. It has been used by a dozen university around the world to build their own versions of the 1/10 autonomous cars. PI Behl's course materials available online have been adapted for teaching CPS and Autonomous Systems courses at UT Austin, and Clemson University. Together with PI Feng, we will develop a new course module which explicitly focuses on trust and behavior modeling of drivers. 
\end{enumerate}

All the PIs reside within the Cyber-Physical Systems Link Lab at UVA and will jointly work together towards creating a template for CPS education centered around behavior guided autonomy, and autonomous vehicles. 

\subsection{K-12 outreach Impacts:}
Co-PI Heydarian has previously helped to organize several outreach programs and workshops (Iridescent, C-TECH2 Workshop, NASA INSPIRE Workshop) promoting STEM studies to underrepresented and minority students in k-12. His research team is currently collaborating with the “Girls Can Change the World with Science” outreach program to inspire young female students across Virginia elementary and middle schools to consider futures in science and engineering fields. He will use the findings of this research to demonstrate how some of our engineering-problems can only be solved by integrating engineering fields (e.g., computer science, system engineering) and other domain(s) together (i.e., psychology). 



\subsection{Graduate/Undergraduate Students and Outreach Effort:}
The PIs are committed to recruiting and nurturing minority and local high-school students by actively participating in local programs such as Women in Computer Science (WICS),  Open-house visitation days, and 1-on-1 career-related advising.
Every year, for the past 3 years, PI Behl has been organizing the International F1/10 Autonomous Racing Competition, where teams from all over the build 1/10 scale cars using open source instructions on how to build, drive, and race these vehicles in a battle of algorithms. The competitions are held in conjunction with a premier venue such as CPS Week (2018), SenSys (2017), and ES-Week (2016, and 2018 Fall). In additional to the organizing the competition, the PI has regularly held F1/10 tutorials to teach undergraduates, and graduate students about autonomy. 
PI Feng has advised over 10 undergraduate students for independent studies or capstone research. She is currently funding two undergraduate students as summer research interns. She organized an N$^2$Women Luncheon at the CPS Week 2015, and is co-organizing a Mentoring Workshop at FLoC 2018.

\subsection{Dissemination Impacts:}
The findings, results, and datasets generated from the proposed testbed will become available to researchers, where they can test different methods and form scientific comparisons of different approaches. We will release general findings, developed software interfaces, algorithms and models via the Web so the interested community (academia and industry) can use and implement our tools and techniques in their work. The scientific community will be made aware of the research project and opportunities for collaboration through the project and Link-Lab webpages, seminars and conferences such as xxxx and SenSys, and mailing lists of targeted communities such as the Society for Personality and Social Psychology, xxxxxxx %add whatever else that is needed
We also plan to disseminate our results in industrial venues such as industry-focused conferences and workshops (e.g., TED, Siggraph, xxxxxxxx).



PI Behl will maintain a website dedicated to the research focus and publish all results, presentations, and videos there, consistent with the Data Management Plan. 
In addition, the PIs are regularly invited by Toyota, US DoT, Virginia Department of Transportation (VDoT), to give invited talks to the AV community.
PI Behl serves on the editorial board of the SAE International Journal of Connected and Automated Vehicles.
This research will lead to the public release of a number of models, data-sets, and tools:
\begin{enumerate}[itemsep=0pt,parsep=0pt,topsep=4pt,leftmargin=0.4in]
    \item Model and data-set library for driving behaviors and emotions - both from simulated and real world experimentation. 
    \item Annotated training data for learning natural language explanations from multimodal autonomous vehicle sensor data.
    \item A UI toolkit with Unity elements for designing feedback for autonomous vehicles - with user study statistics assigned to each element in the kit.
    \item Github repository for code and tutorials related to deep explanations neural networks, and control synthesis algorithms. 
\end{enumerate}