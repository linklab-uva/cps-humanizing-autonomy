%!TEX root = main.tex


\section{Team}
\label{sec:team}
Nicola Bezzo is an Assistant Professor in the Department of Systems and Information Engineering and the Department of Electrical and Computer Engineering at the University of Virginia and a member of the Link Lab. Prior to joining UVa in January 2016, he was a Postdoctoral Researcher at the PRECISE Center, in the Department of Computer and Information Science at the University of Pennsylvania (UPenn) where he worked on topics related to robotics and cyber-physical systems security. He received a Ph.D. degree in Electrical and Computer Engineering from the University of New Mexico where he focused on the development of theories for motion planning of aerial and ground robotic systems under communication constraints. Prior to his Ph.D. he received both M.S. and B.S. degrees in Electrical Engineering with honors (summa cum laude) from Politecnico di Milano, Italy.

Nicola Bezzo's expertise falls under the control and planning of autonomous mobile robots. His research focuses on the design of resilient controller and adaptive planning techniques to improve safety, security, and performance of autonomous aerial and ground vehicles in uncertain situations and environments.  Some of his recent work~\cite{bezzo2016uav,yel2017replanning} focuses on the problem of self-triggered control and planning of autonomous aerial vehicles in cluttered and unstructured environments to minimize energy and computation resources while guaranteeing safety (i.e., nothing bad will ever happen) and liveness (i.e., something good will eventually happen) constraints. In~\cite{bezzo2016stochastic,bezzo2014attack,elnaggar2017safe} he developed resilient state estimators and controllers for autonomous vehicles operations under malicious cyber-attacks. Finally work in~\cite{bezzo2014mech,bezzo2012router,bezzo2011routerswarm} considers the coordination of swarms of heterogeneous robotic systems under sensing, communication, and energy constraints. He has received several awards including the 2016 Robotics and Automation Magazine Best Paper Award, the Best Paper Award at the 2014 CPSWeek International Conference on Cyber-Physical Systems, and the 2010 Gold Medal from the Politecnico School of Engineering.
\lu{Lu adds bio}
Lu Feng is an Assistant Professor at the Department of Computer Science and Department of Systems and Information Engineering at the University of Virginia. She is also a member of the Link Lab - the center of research excellence in Cyber-Physical Systems at the University of Virginia. Previously, she was a postdoctoral fellow at the University of Pennsylvania and received her PhD in Computer Science from the University of Oxford. Her research focuses on assuring the safety, trustworthiness and performance of cyber-physical systems, drawing on formal methods, machine learning and control. She has received several awards including NSF CISE CRII Award, James S. McDonnell Foundation Postdoctoral Fellowship, Rising Stars in EECS, UK Engineering and Physical Sciences Research Council Scholarship, and Cambridge Trust Scholarship.

Dr. Feng's contributions are on temporal logic planning for autonomous robots. Her recent work has focused on synthesizing control protocols for robots that interact with human operators~\cite{feng2016synthesis,feng2015controller} and providing counterexamples as diagnostic information for robotic planning~\cite{feng2018cex,feng2016human}.
She has also develop novel learning-based approaches for the compositional reasoning of probabilistic models with respect to temporal logic specifications~\cite{feng2013learning,feng2011learning,feng2011automated,feng2010compositional}.

Co-PI Heydarian is an Assistant Professor in the Departments of Civil and Environmental Engineering and Systems and Information Engineering at the University of Virginia as well as the Link Lab. He has background and experience in developing human-behavior detection and modeling techniques for human-cyber physical systems, and formalizing systematic approaches for representation and communication techniques.  

\madhur{Madhur adds bio}