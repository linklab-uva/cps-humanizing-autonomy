
Data generated as part of this project, including research manuscripts and technical reports, instrumentation code, modeling code, training data, model descriptions, etc. will be managed using existing University of Virginia infrastructure for administering and maintaining digital research, with automatic nightly back up of source-code and documentation repositories), at no additional cost to the project. Some of the resources and data management practices are already in place, and being utilized by the PI in his research group. We will adopt and extend these practices for short-term data collection, retention and management. 
The PI, Madhur Behl, will have responsibility for coordinating and directing the retention and sharing of data generated through the proposed research activities. Because this work is collaborative across five PIs, many of the data management activities during the project period will reside with the coPIs and will be completed in adherence with NSF and university policies. 
A project website will be established as the main point of access to the generated artifacts including data, source code, publications resulting from the project, and the primary results obtained within the project. This project website will also serve to advertise undergraduate and graduate research opportunities. In addition, we will use Cyber-Physical Systems Virtual Organization (CPS-VO) to disseminate artifacts and information to the CPS/CISE community. 

\subsection*{Expected Data}
An outline of the data expected to be generated by the project is as follows:
\begin{enumerate}
 \item {\em Human driving profile data}
    \begin{description}
    
    \item[Data Collection Methods: ] Our driving behavior data collection methods include observations, interviews, surveys, and driving tests. We will obtain consent from all the participants. The driving experimentson the full scale simulator will be carried out in a safe, and private space in the PIs lab. Any observation notes will not use the participants’ names or other identifying information; instead the information collected will be anonymous.
    
    \item[Data Collection Tools: ] We will employ, photography, audio recordings, and video recordings to collect data. Our IRB protocol will explain in detail what we will record and justify the necessity for using the recording device. In our consent form we will clearly include a section informing the participants that we will be using a recording device. We will make provisions to destroy the recorded materials should the participant decide to withdraw from the study.
    
    \end{description}

 \item {\em Modeling algorithms and code:}
A second major code artifact of our project will be the modeling algorithms and toolchains used to build our scenario models. We will publish the mathematical formulations of these techniques for archival by the publisher, and will maintain implementations of these techniques ({\em i.e.}, code) in version control. Concurrent with the publication of these techniques, we will release code implementing them using open source licenses.
 
\end{enumerate}

\subsection*{Data Retention}
During the project, the data associated with individual tasks will be stored by the PI using his laboratory's local servers and other resources. At regular intervals (quarterly), all the project related data will be copied and archived using the University of Virginia's digital repository. Participant data will be protected by carrying out the following measures. The team will encrypt the data and restrict access with access-level certification so that sensitive information will only be available to authorized personnel and used specifically for research purposes. All human subjects’ data will be handled in accordance with the restrictions of UVA’s Institutional Review Board (IRB), which dictates the appropriate standards for protecting privacy and maintaining confidentiality of respondents. All participants will be informed transparently what data will be collected, and how these data will be reported and used. Note that any sensitive aspects (e.g., concerning human subject data), which may be provided to NSF program officers, may be withheld from public access. Conforming with IRB rules, human-subject data will only be held in anonymized form and will only be released according to IRB procedures. The PI and the research team will be in compliance with all NSF and university policies on research conduct. UVA policies govern the protection of human subjects in any research conducted at UVA, with UVA facilities, or by UVA faculty, staff, or students. Following completion of the project, the data and artifacts emerging from it will be stored for at least 5 years, and after that as long as the external website (or its successor) is maintained. 

\subsection*{Data Formats, Short-term storage and dissemination}
All the data described above and its accompanying documentation will be incrementally made accessible to researchers and the general public as mentioned before.  All the computer codes will be implemented and made accessible to the scientific community in the form live web-tools. Selected source code, associated input/output files and documentation will also be released as the project and this computational modeling technology matures. The researchers retain rights to access and utilize data in whatever format but will not limit requester’s ability to re-use or re-distribute processed data or materials. The data will be deposited in established repositories, for example the UVA institutional repository Libra. Libra is an open repository with public access. Therefore, care will be taken to ensure confidential and sensitive data are not shared through Libra. We reserve the right to delay release of project data for a period of time to allow for publication of research results. This period will not exceed five years following the project end date.

\subsection*{Long-term Data Storage and Preservation of access}
The UVA Libra system provides servers, backup procedures, and other policies to minimize the change of data loss. In accordance with the University of Virginia policy RES-002, “Policy: Laboratory Notebook and Recordkeeping,” the data will be preserved for a minimum of five years upon completion of the project. However the current preservation plan for Libra will be to preserve the data indefinitely.  The Libra backup plan provides for data redundancy including off-site storage.  If the Principal Investigator resigns from the university, the department chairperson for the lead department will become the custodian of the data and will assume all the responsibilities for data management, control and dissemination on behalf of the University of Virginia.


\subsection*{Policies and Provisions for Reuse and Distribution }
The research team will share the research data, wherever appropriate, with the general public through Internet access (including social media), news articles, or reports. The university will regulate this public access in order to protect privacy and address any confidentiality concerns, as well as to respect any personal, proprietary or intellectual property rights. The research team will consult with the university’s legal office to address any concerns on a case-by-case basis, if necessary. Terms of use will include requirements of attribution along with disclaimers of liability in connection with any use or distribution of the research data, which may be conditioned under some circumstances.