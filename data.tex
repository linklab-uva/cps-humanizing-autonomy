
Data generated as part of this project, including instrumentation code, modeling code, training data, model descriptions, etc. will be managed using existing University of Virginia infrastructure for administering and maintaining digital research, with automatic nightly back up of source-code and documentation repositories), at no additional cost to the project. Some of the resources and data management practices are already in place, and being utilized by the PI in his research group. We will adopt and extend these practices for short-term data collection, retention and management. 

The PI, Madhur Behl, will have responsibility for coordinating and directing the retention and sharing of data generated through the proposed research activities. Because this work is collaborative across five PIs, many of the data management activities during the project period will reside with the coPIs and will be completed in adherence with NSF and university policies. 

\subsection*{Expected Data}
An outline of the data expected to be generated by the project is as follows:
\begin{enumerate}
 \item {\em Human driving profile data}
\textbf{Add explanation and address IRB}
 \item {\em Modeling algorithms and code:}
A second major code artifact of our project will be the modeling algorithms and toolchains used to build our scenario models. We will publish the mathematical formulations of these techniques for archival by the publisher, and will maintain implementations of these techniques ({\em i.e.}, code) in version control. Concurrent with the publication of these techniques, we will release code implementing them using open source licenses.
 
\item {\em User interface design data:}
\textbf{Add explanation}

\end{enumerate}

\subsection*{Period of Data Retention}
During the project, the data associated with individual tasks will be stored by the PI using his laboratory's local servers and other resources. At regular intervals (quarterly), all the project related data will be copied and archived using the University of Virginia's digital repository. 

\subsection*{Data Formats, Short-term storage and dissemination}
All the data described above and its accompanying documentation will be incrementally made accessible to researchers and the general public as mentioned before.  All the computer codes will be implemented and made accessible to the scientific community in the form live web-tools. Selected source code, associated input/output files and documentation will also be released as the project and this computational modeling technology matures. The researchers retain rights to access and utilize data in whatever format but will not limit requester’s ability to re-use or re-distribute processed data or materials. The data will be deposited in established repositories, for example the UVA institutional repository Libra. Libra is an open repository with public access. Therefore, care will be taken to ensure confidential and sensitive data are not shared through Libra. We reserve the right to delay release of project data for a period of time to allow for publication of research results. This period will not exceed three years following the project end date.

\subsection*{Long-term Data Storage and Preservation of access}
The UVA Libra system provides servers, backup procedures, and other policies to minimize the change of data loss. In accordance with the University of Virginia policy RES-002, “Policy: Laboratory Notebook and Recordkeeping,” the data will be preserved for a minimum of five years upon completion of the project. However the current preservation plan for Libra will be to preserve the data indefinitely.  The Libra backup plan provides for data redundancy including off-site storage.  If the Principal Investigator resigns from the university, the department chairperson for the lead department will become the custodian of the data and will assume all the responsibilities for data management, control and dissemination on behalf of the University of Virginia.

