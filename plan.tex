\section{Project Management and Collaboration Plan}
\label{sec:plan}
This is an ambitious proposal which addresses challenging problems in autonomous vehicles, but we have an excellent team with broad and complementary skills, the facilities to support it, and a strong basis to start from.
The proposed work will be performed by the PIs Behl, Heydarian, Bezzo, and Feng, and 4 graduate students.

\subsection{Roles and responsibilities:}


\subsection{Project tasks and time-line:}

The timeline for the proposed work  is shown in the table~\ref{fig:gantt}.

\begin{figure}
    \centering
    \includegraphics[width=0.8\columnwidth]{figures/gantt_plan.png}
    \caption{Description of project tasks, roles, and timeline for the proposed research}
    \label{fig:gantt}
\end{figure}


\subsection{Risks and mitigation plans:}

Our excellent team and access to various automotive cyber-physical systems test-beds gives us the confidence that the proposed research can be carrier out without any major hassles. While we expect some difficulties may arise along the way, these would not sabotage the effort. Examples include component failures (these can be replaced), slow runtimes of software (this will be optimized), and realism of the simulated scenarios (these will be validated by making the same subject drive both in simulation and real world and compare the behavior models). The proposed work involves the use of human subjects for social and behavioral studies and that may be perceived as a risk but the team is experienced with working with IRB.

\subsection{Results from Prior NSF Support}
\label{subsec:prior}

Madhur Behl: (NSF-1735587); $\$2,499,238$; 09/1/2017-08/31/2021 Title: CRISP Type 2: dMIST: Data-driven Management for Interdependent Stormwater and Transportation Systems. Role: Co-PI. Intellectual Merit: To create a novel decision support system denoted dMIST (Data-driven Management for Interdependent Stormwater and Transportation Systems) to improve management of interdependent transportation and stormwater infrastructure systems. Behl's role on this project is the development of a novel modeling and control framework called Data Predictive Control (DPC) for assisting decision makers in understanding and managing interdependent critical infrastructure systems. Broader impact: The research is intended to have broad impact related to national economic and security interests due to its focus on sea level rise. Paper submitted to $11^{th}$ International Conference on Urban Drainiage Modeling. 



\vspace{4pt}%\newline
\noindent{Nicola Bezzo:}

\vspace{4pt}%\newline
\noindent{Lu Feng:}


Co-PI Heydarian does not prior NSF funding. 
