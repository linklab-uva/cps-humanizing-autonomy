\section{Project Management and Collaboration Plan}
\label{sec:plan}
This is an ambitious proposal which addresses challenging problems in autonomous vehicles, but we have an excellent team with broad and complementary skills, the facilities to support it, and a strong basis to start from.
The proposed work will be performed by the PIs Behl, Heydarian, Bezzo, and Feng, and 4 graduate students.

\subsection{Roles and responsibilities:}

\textbf{Cognitive, Behavioral, and Social Sensing:} Co-PI Heydarian’s research focuses on integrating user-centered considerations (i.e., human behavior and emotional models) into the design and operation of automated systems (Human-in-the-Loop CPS). In his research, Heydarian has examined the impact of human preferential and behavioral differences on building energy consumption through a bottom-up approach, where users’ behavioral and preferential information were collected and integrated into building performance simulations for design and operational improvements and decision making (CITE). Through conducting a number of experimental studies (over 700 consenting participants), he has examined how psychological theories such as “default settings” or “personality traits” can influence users preferences and interaction with building systems (CITE). In the proposed research project, Heydarian will lead the effort for the deployment of sensing and modeling technologies/techniques for understanding the impact of environmental, emotional, and social factors on driver and passenger(s) behaviors. 




\subsection{Project tasks and time-line:}

The timeline for the proposed work  is shown in the table~\ref{fig:gantt}.

\begin{figure}
    \centering
    \includegraphics[width=0.8\columnwidth]{figures/gantt_plan.png}
    \caption{Description of project tasks, roles, and timeline for the proposed research}
    \label{fig:gantt}
\end{figure}


\subsection{Risks and mitigation plans:}

Our excellent team and access to various automotive cyber-physical systems test-beds gives us the confidence that the proposed research can be carrier out without any major hassles. While we expect some difficulties may arise along the way, these would not sabotage the effort. Examples include component failures (these can be replaced), slow runtimes of software (this will be optimized), and realism of the simulated scenarios (these will be validated by making the same subject drive both in simulation and real world and compare the behavior models). The proposed work involves the use of human subjects for social and behavioral studies and that may be perceived as a risk but the team is experienced with working with IRB.

\subsection{Results from Prior NSF Support}
\label{subsec:prior}

Madhur Behl: (NSF-1735587); $\$2,499,238$; 09/1/2017-08/31/2021 Title: CRISP Type 2: dMIST: Data-driven Management for Interdependent Stormwater and Transportation Systems. Role: Co-PI. Intellectual Merit: To create a novel decision support system denoted dMIST (Data-driven Management for Interdependent Stormwater and Transportation Systems) to improve management of interdependent transportation and stormwater infrastructure systems. Behl's role on this project is the development of a novel modeling and control framework called Data Predictive Control (DPC) for assisting decision makers in understanding and managing interdependent critical infrastructure systems. Broader impact: The research is intended to have broad impact related to national economic and security interests due to its focus on sea level rise. Paper submitted to $11^{th}$ International Conference on Urban Drainiage Modeling. 



\vspace{4pt}%\newline
\noindent{Nicola Bezzo:}

\vspace{4pt}%\newline
\noindent{Lu Feng:}
Co-PI of NSF grant CNS-1739333 ``CPS: Medium: Safety-Critical Wireless Mobile Systems'', \$800,000, 9/1/2017-8/31/2020. \textbf{Intellectual Merits}: This project aims to develop new techniques for safe wirelessly coordinated mobility.
It develops a framework for joint modeling and analysis of motion and communication in order to find provably safe coordination paths. 
\textbf{Broader Impacts}: The research will allow mobile systems to realize the performance benefits of wireless coordination while preserving the ability to provide provable safety guarantees. 

PI of NSF grant CNS-1755784 ``CRII: CPS: Cognitive Trust in Human-Autonomous Vehicle Interactions'', \$175,000, 4/1/2018-3/31/2020.
\emph{Intellectual Merits}: The research is to develop new formal specification and verification methods for formally expressing and reasoning about trust in human-autonomous vehicle interactions.
\emph{Broader Impacts}: The research will create new techniques for assisting in the design of safe and trustworthy autonomy into future vehicles.


Co-PI Heydarian does not prior NSF funding. 
